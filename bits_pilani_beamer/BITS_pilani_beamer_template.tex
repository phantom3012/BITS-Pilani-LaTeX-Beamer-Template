\documentclass{beamer}

\mode<presentation> {
\usetheme{bits}
%\setbeamertemplate{Fluid Mechanics} % To remove the footer line in all slides uncomment this line
% \setbeamertemplate{footline}{} % To replace the footer line in all slides with a simple slide count uncomment this line
\setbeamertemplate{navigation symbols}{} % To remove the navigation symbols from the bottom of all slides uncomment this line
}
\usepackage{graphicx} % Allows including images
\usepackage{booktabs} % Allows the use of \toprule, \midrule and \bottomrule in tables
\usepackage[english]{babel}
\usepackage{listings,multimedia}
\usepackage{multimedia}
\usepackage{ulem}
\usepackage{chemfig}
\usepackage{amsmath}
\usepackage{mathtools}
\usepackage{commath}
\usepackage{float}
\usepackage{caption}
\usepackage{multirow}
\usepackage{amssymb}
\usepackage[FIGTOPCAP,hang,nooneline]{subfigure}
\usepackage{tikz}
\usepackage[labelformat=empty]{caption}
\renewcommand{\thesubfigure}{}
\newcommand{\lmath}[1]{$\mathrm{#1}$}
\newcommand{\lbmath}[1]{$\mathbf{#1}$}

\usepackage{csquotes}
\usepackage[backend=biber]{biblatex}
\addbibresource{ref.bib}
\setbeamertemplate{bibliography item}{\insertbiblabel}

\newenvironment{descrsf}[1]
  {\begin{list}{}{\renewcommand{\makelabel}[1]{\textsf{##1}\hfil}
                  \setlength{\itemsep}{0.5em}
                  \setlength{\parsep}{0pt}
                  \settowidth{\labelwidth}{\textsf{#1}}
                  \setlength{\labelsep}{10pt}
                  \setlength{\leftmargin}{\labelwidth}
                  \addtolength{\leftmargin}{\labelsep}
                  \providecommand{\descriptionlabel}[1]%
                  {\hspace{\labelsep}\textsf{#1}}
                 }
  }
  {\end{list}}
  \newcommand\blfootnote[1]{%
  \begingroup
  \renewcommand\thefootnote{}\footnote{#1}%
  \addtocounter{footnote}{-1}%
  \endgroup
}

\newcommand{\addlogo}{\hfill\includegraphics[width=.2\textwidth,height=0.05\paperheight,trim=0 -0.65cm 1.5cm 2cm]{brand_strip_logo.png}}
\newcommand{\bitstitle}[1]{\makebox[\framewidth]{#1 \addlogo}}

%----------------------------------------------------------------------------------------
%	TITLE PAGE
%----------------------------------------------------------------------------------------
\usefonttheme{serif}
\title[Firmware verification for wBMS]{\small{Firmware verification for Automotive Wireless Battery Monitoring Systems \\[1ex]}} % The short title appears at the bottom of every slide, the full title is only on the title page
\author[Sai Kartik]{\begin{tabular}{r@{ }l} 
  Author:       & Sai Kartik (2020A3PS0435P) \\
  Manager:  & Mr. Abhinandan Subbarao\\
  Professor:    & Dr. Sathisha Shet K \\
  PS Station: & Analog Devices, India
\end{tabular}} % Your name
\institute[BPPC]{BITS Pilani, Pilani Campus} % Your institution as it will appear on the bottom of every slide, may be 
 \date{September 23,2023} % Date, can be changed to a custom date

\begin{document}
\setbeamertemplate{footline}{}
\begin{frame}
  \titlepage
\end{frame}

\setbeamertemplate{footline}[infolines theme]

\begin{frame}
  \frametitle{\bitstitle{Aim}}
  The objective of this project is to validate the firmware on the battery monitor sensors in order to ensure its compliance with established functional safety criteria for a wireless battery monitoring system used in an automotive environment.
\end{frame}

\begin{frame}
  \frametitle{\bitstitle{Introduction}}
  \begin{itemize}
    \item Growing need for modern cockpit electronics: Result of the industry's shift toward electric mobility
    \item Also a growing need for highly safe systems to monitor various components of the automotive in concern
    \item The battery is the most important component in any electric vehicle and requires constant monitoring. Especially considering most of the vehicles have packs with Li-ion substrates. The damage can be catastrophic if not monitored closely and properly \pause
    \item ADI provides various wBMS solutions that can help address concerns regarding the same
  \end{itemize}
\end{frame}

\begin{frame}
  \frametitle{\bitstitle{Product Walkthrough}}
  \begin{itemize}
    \item Hardware setup
          \begin{itemize}
            \item Sensors
            \item Monitors
            \item Connected wirelessly through specific connection protocols
            \item Specialised library designed to work with client microcontollers to collect data
            \item Various safety guidelines agreed upon by all parties involved. These will be implemented on all the hardware components involved
          \end{itemize}
    \item Software setup
          \begin{itemize}
            \item Relevant firmware that runs on each controller present
            \item Using the software that runs on the sensors, various functional safety capabilities can be tested
                  \begin{itemize}
                    \item Either inject erroneous instructions into the sensor, or corrupt the sensor output
                    \item Compare this to a set baseline to evaluate the performance
                  \end{itemize}
            \item Various frameworks that utilise this concept have been developed and tested in the field
          \end{itemize}
  \end{itemize}
\end{frame}

\begin{frame}
  \frametitle{\bitstitle{Testing Methodologies}}
  \begin{itemize}
    \item Original testing procedure was implemented using python scripts with a combination of scripting techniques and OOP concepts
          \begin{itemize}
            \item Problems: Difficult to maintain, non-flexible framework
          \end{itemize}
    \item Solution: Newer test frameworks being developed are utilising OOP concepts to aid in easy maintenance of the codebase and retain flexibility of requirement-based changes
    \item Use APIs exposed by ADI's internal software to directly communicate with the system to evaluate/baseline performance. Also helps in keeping the codebase modular
  \end{itemize}
\end{frame}

\begin{frame}
  \frametitle{\bitstitle{Work Progress}}
  \begin{itemize}
    \item Completed understanding of various hardware and software components of the system
    \item Analysed various faults with the previous testing methodologies and how they can be improved
    \item Basic implementation of the newer framework using OOP concepts has been implemented. With minor modifications, the test framework will be easy to use for all possible test cases
    \item Modularised code to a large extent
    \item (Timelines for specific events updated in the diary)
  \end{itemize}
\end{frame}

\begin{frame}
  \frametitle{\bitstitle{Future Work}}
  \begin{itemize}
    \item Complete implementation of said OOP framework
    \item Work on automation of testing the various cases possible \cite{jenkins}
    \item Extend the framework to work with various sensor families
  \end{itemize}
\end{frame}

\begin{frame}
  \frametitle{\bitstitle{Product Innovation}}
  A major innovative aspect of this project is to purpose various freely available tools (like python/pytest) \cite{pythondocs,pytest} to generate automatic reports on testing various firmware which is very specific to a certain use case
\end{frame}

\begin{frame}
  \frametitle{\bitstitle{References\footnote[frame]{Other documents regarding specific hardware/software architecture are for internal use only and cannot be shared as open sources/references}}}
  \printbibliography
\end{frame}

\end{document}