\documentclass{beamer}

\mode<presentation> {
\usetheme{bits}
%\setbeamertemplate{Fluid Mechanics} % To remove the footer line in all slides uncomment this line
% \setbeamertemplate{footline}{} % To replace the footer line in all slides with a simple slide count uncomment this line
\setbeamertemplate{navigation symbols}{} % To remove the navigation symbols from the bottom of all slides uncomment this line
}
\usepackage{graphicx} % Allows including images
\usepackage{booktabs} % Allows the use of \toprule, \midrule and \bottomrule in tables
\usepackage[english]{babel}
\usepackage{listings,multimedia}
\usepackage{multimedia}
\usepackage{ulem}
\usepackage{chemfig}
\usepackage{amsmath}
\usepackage{mathtools}
\usepackage{commath}
\usepackage{float}
\usepackage{caption}
\usepackage{multirow}
\usepackage{amssymb}
\usepackage[FIGTOPCAP,hang,nooneline]{subfigure}
\usepackage{tikz}
\usepackage[labelformat=empty]{caption}
\renewcommand{\thesubfigure}{}
\newcommand{\lmath}[1]{$\mathrm{#1}$}
\newcommand{\lbmath}[1]{$\mathbf{#1}$}

\usepackage{csquotes}
\usepackage[backend=biber]{biblatex}
\addbibresource{ref.bib}
\setbeamertemplate{bibliography item}{\insertbiblabel}

\newenvironment{descrsf}[1]
  {\begin{list}{}{\renewcommand{\makelabel}[1]{\textsf{##1}\hfil}
                  \setlength{\itemsep}{0.5em}
                  \setlength{\parsep}{0pt}
                  \settowidth{\labelwidth}{\textsf{#1}}
                  \setlength{\labelsep}{10pt}
                  \setlength{\leftmargin}{\labelwidth}
                  \addtolength{\leftmargin}{\labelsep}
                  \providecommand{\descriptionlabel}[1]%
                  {\hspace{\labelsep}\textsf{#1}}
                 }
  }
  {\end{list}}
  \newcommand\blfootnote[1]{%
  \begingroup
  \renewcommand\thefootnote{}\footnote{#1}%
  \addtocounter{footnote}{-1}%
  \endgroup
}

\newcommand{\addlogo}{\hfill\includegraphics[width=.2\textwidth,height=0.05\paperheight,trim=0 -0.65cm 1.5cm 2cm]{brand_strip_logo.png}}
\newcommand{\bitstitle}[1]{\makebox[\framewidth]{#1 \addlogo}}

%----------------------------------------------------------------------------------------
%	TITLE PAGE
%----------------------------------------------------------------------------------------
\usefonttheme{serif}
\title[Firmware verification for wBMS]{\small{Firmware verification for Automotive Wireless Battery Monitoring Systems \\[1ex]}} % The short title appears at the bottom of every slide, the full title is only on the title page
\author[Sai Kartik]{\begin{tabular}{r@{ }l} 
  Author:       & Sai Kartik (2020A3PS0435P) \\
  Manager:  & Mr. Abhinandan Subbarao\\
  Professor:    & Dr. Sathisha Shet K \\
  PS Station: & Analog Devices, India
\end{tabular}} % Your name
\institute[BPPC]{BITS Pilani, Pilani Campus} % Your institution as it will appear on the bottom of every slide, may be 
 \date{December 15,2023} % Date, can be changed to a custom date

\begin{document}
\setbeamertemplate{footline}{}
\begin{frame}
  \titlepage
\end{frame}

\begin{frame}
  \frametitle{\bitstitle{Outline}}
  \tableofcontents
\end{frame}

\setbeamertemplate{footline}[infolines theme]

\section{Aim and Problem statement}
\begin{frame}
  \frametitle{\bitstitle{Aim and Problem statement}}
  \textbf{Aim} \\ This project aims to verify the firmware of battery monitor sensors for a car's wireless battery monitoring system (wBMS).
  \vfill
  \textbf{Problem statement} \\ Use automation concepts to test the software and deliver it to customers quickly and efficiently without bugs.
\end{frame}

\section{Main Objectives}
\begin{frame}
  \frametitle{\bitstitle{Main Objectives}}
  \begin{itemize}
    \item To identify the testcases to be executed on wBMS system
    \item To write scripts to perform manual testing of all tests
    \item To automate the running of the test suite and generation of test report
  \end{itemize}

\end{frame}

\section{Design Methodologies}
\begin{frame}
  \frametitle{\bitstitle{Design Methodologies}}
  \begin{itemize}
    \item To perform manual testing, we flash the firmware onto the respective devices with JLink lite debuggers % insert ref 40
    \item Download files to the gateway and the nodes to faciliate communication between them % insert ref 22
    \item Configure the front-end application (GUI) to control the network and test the functionality.
    \item Ensure the setup is RF shielded fairly well
  \end{itemize}

\end{frame}

\section{Implementation}

\subsection{High level implementation}
\begin{frame}
  \frametitle{\bitstitle{High level implementation}}
  The software test cycle (STLC) mainly consists of 4 major steps to go through:
  \begin{itemize}
    \item Test planning \pause
    \item Test case development \pause
    \item Test environment setup \pause
    \item Test execution \pause
  \end{itemize}

\end{frame}

\subsubsection{Test Planning}
\begin{frame}
  \frametitle{\bitstitle{Test Planning}}
  \begin{itemize}
    \item This step is the most significant part of software testing where the required testing strategies are created. \pause
    \item It is typically the team lead's/manager's role to establish the project cost and efforts required. \pause % insert ref 32 
    \item This phase begins once the requirement collection phase has been completed.
    \item The major outcome of this phase is the finalised test plan/strategy which has to be adhered to. \pause
  \end{itemize}

\end{frame}

\subsubsection{Test Case Development}
\begin{frame}
  \frametitle{\bitstitle{Test Case Development}}
  \begin{itemize}
    \item Once a strategy of the tests to be performed is outlined, the required data for it is gathered. \pause
    \item This data is organised to fit various test cases to ensure coverage of all possible scenarios. \pause
    \item Once the design of individual test cases is complete, each test case is linked in a chain in the Responsibility Traceability Matrix. \pause %insert ref 33
  \end{itemize}

\end{frame}


\begin{frame}
  \frametitle{\bitstitle{Conclusion}}

\end{frame}

\setbeamertemplate{footline}{}
\begin{frame}
  \frametitle{\bitstitle{References\footnote[frame]{Other documents regarding specific hardware/software architecture are for internal use only and cannot be shared as open sources/references}}}
  % \printbibliography
\end{frame}

\end{document}